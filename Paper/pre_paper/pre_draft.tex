%コンパイル方法: opt+cmd+b → opt+cmd+v
\RequirePackage{plautopatch}

\documentclass[a4paper, 10pt]{ltjsarticle}


% マージン設定
\usepackage[top=20mm, bottom=20mm, left=20mm, right=20mm]{geometry}

% LuaLaTeX用日本語対応パッケージ
\usepackage{luatexja}
\usepackage{luatexja-fontspec}

% 必要なパッケージ
\usepackage{fontspec}
\usepackage{titlesec}
\usepackage{graphicx}
\usepackage{amsmath}
\usepackage{amssymb}
\usepackage[hidelinks]{hyperref}
\usepackage[english, japanese]{babel}
\usepackage{multicol} % 二段組用パッケージ
\usepackage{indentfirst}
\usepackage{tikz} % カスタム点線用
\usepackage{authblk} % 著者・所属パッケージ
\usepackage{here}
\usepackage{caption}
\usepackage{bookmark}

% \setmainfont[Ligatures=TeX]{Times New Roman}
% \setmainjfont[BoldFont=MS Gothic]{MS Mincho}

\renewcommand{\baselinestretch}{0.95}
\renewcommand{\labelenumi}{(\arabic{enumi})}

% セクション見出しのカスタマイズ
\titleformat{\section}
  {\fontsize{10pt}{10pt}}
  {\thesection.}
  {1em}{}

\titleformat{\subsection}
  {\fontsize{10pt}{10pt}}
  {\thesubsection}
  {1em}{}

\titleformat{\subsubsection}
  {\fontsize{10pt}{10pt}}
  {\thesubsubsection}
  {1em}{}

  \setlength{\parindent}{1em}
% \setlength{\belowcaptionskip}{1em} % キャプション下の余白を -10pt に設定


%section前に余白を作る
\titlespacing*{\section}{0em}{1em}{0em}
\titlespacing*{\subsection}{0em}{1em}{0em}
\titlespacing*{\subsubsection}{0em}{1em}{0em}

\pagestyle{empty}


\begin{document}

\setlength{\columnsep}{7.5mm}

\twocolumn[
    \begin{center}
        {\vspace{-1em}}

        {\fontsize{15pt}{15pt}\selectfont{アドホックネットワークについての研究(仮)}}

        {\vspace{1.5em}}

        {\fontsize{13pt}{13pt}\selectfont{}}
    \end{center}



    \begin{flushright}
      {\fontsize{11pt}{11pt}\selectfont{T5-17 末廣隼人\\}}
      {\fontsize{11pt}{11pt}\selectfont{指導教員 髙﨑和之}}
    \end{flushright}

    \vspace{1em}

    \thispagestyle{empty}
]

\section{研究背景}
% 自然災害、特に地震発生時に基地局の倒壊やネットワーク障害が生じたときに、
% その間に助けを求める人達の不安を少しでも拭うために一時的なネットワーク、
% アドホックネットワークの構築を行い少しでも多くの情報を共有できることを目的として研究を行った。\\
% 具体的には、被災エリアの中心に主となるノードを設置、その周りに携帯端末が保有するBluetoothなどでアドホックネットワークの構築おこうなう。
% このとき、全ての端末をアドホックに使用してしまうと、ルーティングが煩雑distribution化してしまうしまうため、
% アルゴリズムでメインとして使用する端末と接続が切れてしまった時の補助端末に分ける。その方法を第3章で述べる。

% 日本では多くの自然災害が発生しており、自然災害の中で一番恐れらている災害が地震であった。
% 昨年の石川県能登半島で発生した大地震では多くの死傷者がでてしまい、甚大な被害を被った。
% この災害では特に高齢者の人口が多く占めており、迅速な避難が難しかったり、

\section{理論}
\subsection{アドホックネットワーク}
アドホックネットワークとは、中央の管理者や既存のインフラストラクチャ(ルータやアクセスポイント等)を介さずに、端末(以降ノードと呼ぶ)同士が直接通信を行う一時的なネットワークのことである。
遠くのノードと通信を行うとき、隣接する他ノードを中継機として利用し、バケツリレーのようにデータを送信する「マルチホップ通信」技術を用いて通信を行う。%

\subsection{アドホックネットワークの技術的課題}
\subsubsection{隠れ端末問題}
隠れ端末問題とは、図1のようにノードAとCがノードBに対して通信を行うとき、ノードAとCはお互いの存在が隠れてしまい、
現在誰も通信を行なってないと思い込んで同時にノードBへと通信を行いデータが衝突してし壊れてしまう問題である。%
\begin{figure}[H]
  \centering
  \includegraphics[width=70mm]{hidden_terminal_problem.png}
  \caption{隠れ端末問題}
\end{figure}

\subsubsection{さらし端末問題}
さらし端末問題とは、図2のようにノードAがDと通信を行なっているときノードBは端末Cと通信ができそうだが、
ノードAがDと通信を行なっているため周辺にいる他ノードは通信の抑制がされてしまい、
伝送速度や通信品質の低下が発生してしまう問題である。%
\begin{figure}[H]
  \centering
  \includegraphics[width=70mm]{exposed_terminal_problem.png}
  \caption{さらし端末問題}
\end{figure}

\subsection{ルーティング制御方法}
各ノードと通信を行う際のルーティング方式について述べる。
ルーティング方式には大きく分けてリアクティブ型、プロアクティブ型、ハイブリッド型の3種類がある。
\begin{enumerate}
  \item \label{reactive} リアクティブ型\\通信要求を行った時に近くのノードとその場で経路を作成する。そのため、通信開始までに遅延が生じる。
  \item \label{proactive} プロアクティブ型\\近くのノード間で自身の情報をやりとし経路をあらかじめ作成するルーティング方式である。あらかじめ経路が作成されるため、通信開始までの遅延が短い。
  しかし、定期的にデータのやり取りを行うため電池の消費が早い。
  \item ハイブリッド型\\(\ref{reactive})、(\ref{proactive})の二つを組み合わせたルーティング方式である。
\end{enumerate}
\section{従来手法}
%先輩のシミュレーションについてのべる
\section{提案手法}
\subsection{想定環境}
今回、経路を生成するのにあたりノード数は日本の各都道府県の人口密度に持ち運びが便利なスマートフォンの所有率88.6\%(2021年次の情報)を掛け合わせた数とする。%
用いる都道府県は人口密度が高い地域、低い地域と昨年被災した石川県珠洲市の3つの地域に対してシミュレーションを行う。
\\従来手法では、人口密度が高い地域では$1\mathrm{km}^2$内に存在する全てのノードが通信できるようにカバーされていたが、
人口密度が低い地域ではたまたま
\subsection{提案手法}
\section{結果}

\section{まとめ}


\end{document}
